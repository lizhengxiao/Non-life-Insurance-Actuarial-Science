% Options for packages loaded elsewhere
\PassOptionsToPackage{unicode}{hyperref}
\PassOptionsToPackage{hyphens}{url}
%
\documentclass[
]{book}
\usepackage{amsmath,amssymb}
\usepackage{lmodern}
\usepackage{iftex}
\ifPDFTeX
  \usepackage[T1]{fontenc}
  \usepackage[utf8]{inputenc}
  \usepackage{textcomp} % provide euro and other symbols
\else % if luatex or xetex
  \usepackage{unicode-math}
  \defaultfontfeatures{Scale=MatchLowercase}
  \defaultfontfeatures[\rmfamily]{Ligatures=TeX,Scale=1}
\fi
% Use upquote if available, for straight quotes in verbatim environments
\IfFileExists{upquote.sty}{\usepackage{upquote}}{}
\IfFileExists{microtype.sty}{% use microtype if available
  \usepackage[]{microtype}
  \UseMicrotypeSet[protrusion]{basicmath} % disable protrusion for tt fonts
}{}
\makeatletter
\@ifundefined{KOMAClassName}{% if non-KOMA class
  \IfFileExists{parskip.sty}{%
    \usepackage{parskip}
  }{% else
    \setlength{\parindent}{0pt}
    \setlength{\parskip}{6pt plus 2pt minus 1pt}}
}{% if KOMA class
  \KOMAoptions{parskip=half}}
\makeatother
\usepackage{xcolor}
\usepackage{longtable,booktabs,array}
\usepackage{calc} % for calculating minipage widths
% Correct order of tables after \paragraph or \subparagraph
\usepackage{etoolbox}
\makeatletter
\patchcmd\longtable{\par}{\if@noskipsec\mbox{}\fi\par}{}{}
\makeatother
% Allow footnotes in longtable head/foot
\IfFileExists{footnotehyper.sty}{\usepackage{footnotehyper}}{\usepackage{footnote}}
\makesavenoteenv{longtable}
\usepackage{graphicx}
\makeatletter
\def\maxwidth{\ifdim\Gin@nat@width>\linewidth\linewidth\else\Gin@nat@width\fi}
\def\maxheight{\ifdim\Gin@nat@height>\textheight\textheight\else\Gin@nat@height\fi}
\makeatother
% Scale images if necessary, so that they will not overflow the page
% margins by default, and it is still possible to overwrite the defaults
% using explicit options in \includegraphics[width, height, ...]{}
\setkeys{Gin}{width=\maxwidth,height=\maxheight,keepaspectratio}
% Set default figure placement to htbp
\makeatletter
\def\fps@figure{htbp}
\makeatother
\setlength{\emergencystretch}{3em} % prevent overfull lines
\providecommand{\tightlist}{%
  \setlength{\itemsep}{0pt}\setlength{\parskip}{0pt}}
\setcounter{secnumdepth}{5}
\usepackage{booktabs}
\usepackage{amsthm}
\makeatletter
\def\thm@space@setup{%
  \thm@preskip=8pt plus 2pt minus 4pt
  \thm@postskip=\thm@preskip
}
\makeatother
\ifLuaTeX
  \usepackage{selnolig}  % disable illegal ligatures
\fi
\usepackage[]{natbib}
\bibliographystyle{apalike}
\IfFileExists{bookmark.sty}{\usepackage{bookmark}}{\usepackage{hyperref}}
\IfFileExists{xurl.sty}{\usepackage{xurl}}{} % add URL line breaks if available
\urlstyle{same} % disable monospaced font for URLs
\hypersetup{
  pdftitle={非寿险精算与风险模型},
  hidelinks,
  pdfcreator={LaTeX via pandoc}}

\title{非寿险精算与风险模型}
\author{}
\date{\vspace{-2.5em}2023-05-09}

\begin{document}
\maketitle

{
\setcounter{tocdepth}{1}
\tableofcontents
}
\hypertarget{prep}{%
\chapter{课程简介}\label{prep}}

这是对外经济贸易大学和西南财经大学开设《非寿险精算学》等课程的讲义, 其中使用了其它教材的例子和讲法, 仅供学生内部使用, 不作为公开出版。 鉴于本人水平有限, 错漏之处难免,欢迎指出错误或提出改进意见。

\textbf{\emph{编写人员}}:

\begin{itemize}
\item
  李政宵(对外经济贸易大学保险学院副教授)
\item
  杨亮(西南财经大学金融学院副教授)
\item
  刘坤(对外经济贸易大学 2020 级硕士生)
\item
  刘雅薇(对外经济贸易大学 2020 级硕士生)
\end{itemize}

\hypertarget{ux8bfeux7a0bux5185ux5bb9}{%
\section{课程内容}\label{ux8bfeux7a0bux5185ux5bb9}}

本课程是一门三学分课程(精算与风险管理专业必修、其他专业选修),适合风险管理、保险与精算等相关专业的\textbf{本科高年级学生}参考。

非寿险精算主要内容是\textbf{风险模型}、\textbf{费率厘定}和\textbf{准备金评估}。

\begin{itemize}
\tightlist
\item
  \textbf{风险模型}:风险度量、索赔次数模型、索赔金额模型、累积损失模型
\item
  \textbf{费率厘定}:分类费率厘定、经验费率厘定
\item
  \textbf{准备金评估}:未到期责任准备金、未决赔款准备金、理赔费用准备金评估
\end{itemize}

在教学过程中中,以 \textbf{R 语言} 为编程工具,同时提供了详细的程序代码,方便读者再现完整的编程和计算过程。

\hypertarget{ux5148ux4feeux5185ux5bb9}{%
\section{先修内容}\label{ux5148ux4feeux5185ux5bb9}}

需要概率论与数理统计、高等数学、线性代数、保险学的基础知识

\begin{itemize}
\tightlist
\item
  \textbf{概率论与数理统计}:随机变量、概率分布、中心极限定理
\item
  \textbf{高等数学}:微积分、泰勒公式
\item
  \textbf{线性代数}:矩阵运算
\item
  \textbf{保险学}:基本概念和专业术语
\end{itemize}

\hypertarget{ux6559ux6750ux548cux53c2ux8003ux8d44ux6599}{%
\section{教材和参考资料}\label{ux6559ux6750ux548cux53c2ux8003ux8d44ux6599}}

\begin{itemize}
\item
  Klugman S. A., Panjer H. H., Willmot G. E. Loss models: from data to decisions (5th edition). London: John Wiley \& Sons, 2016.
\item
  孟生旺、刘乐平、肖争艳、高光远, 非寿险精算学(第4版),人民大学出版社,2019年.
\item
  孟生旺, 《风险模型》, 中国人民大学出版社,2022.
\item
  肖争艳,《精算模型》(第三版),中国人民大学出版社,2019.
\end{itemize}

\hypertarget{r-ux8f6fux4ef6ux4e0e-rmarkdown-ux7b80ux4ecb}{%
\section{R 软件与 Rmarkdown 简介}\label{r-ux8f6fux4ef6ux4e0e-rmarkdown-ux7b80ux4ecb}}

\begin{itemize}
\item
  R 软件是用于统计计算和绘图的免费软件环境。
  安装 R 软件的步骤主要分下述几步步:

  \begin{itemize}
  \item
    \href{https://cran.r-project.org/}{安装 R 软件}
  \item
    \href{https://posit.co/download/rstudio-desktop/}{安装 RStudio}。RStudio 是一款 R 语言的 IDE(集成开发环境),操作界面简洁美观。
  \item
    \href{https://mirrors.tuna.tsinghua.edu.cn/CRAN/}{安装 rtools}。rtools 是 R 软件在构建和编译包所需要的编译工具。
  \end{itemize}
\item
  \texttt{Rmarkdown} 是 R 语言环境中提供的 markdown 编辑工具。可以用来创建富有格式和互动性的文档。它可以让用户将代码、文本、图表等内容组合到一个文档中,以生成多种格式的输出,如 HTML、PDF、Word 文档等。\texttt{Rmarkdown} 的作用包括:

  \begin{itemize}
  \item
    \texttt{创建文档}:用户可以使用 Rmarkdown 编写报告、论文、幻灯片、教程、笔记本等各种类型的文档,从而将所有相关信息组织到一个单一的文档中,以方便交流和分享。
  \item
    \texttt{自动生成报告}:用户可以使用 Rmarkdown 自动化报告的生成,通过将代码嵌入文档中,生成数据分析、机器学习等各种类型的报告,以便更好地分享分析结果。
  \item
    \texttt{交互性可视化}:用户可以使用 Rmarkdown 和其它 R 包结合使用,创建交互式可视化的报告,使读者可以通过控制台与图表进行交互,探索数据的不同方面。
  \item
    \texttt{维护数据分析文档}:用户可以使用 Rmarkdown 生成可重复的分析文档,以便在数据分析项目中跟踪分析的演变,并使其更加容易维护。
  \end{itemize}
\end{itemize}

  \bibliography{book.bib,packages.bib}

\end{document}
